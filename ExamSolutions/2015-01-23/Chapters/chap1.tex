\section{Question 1: True-False}\label{sec:q1}    
\begin{enumerate}[label=\alph*]
    \item The total energy of a vehicle returning from space is equally divided over potential and kinetic energy. \textbf{FALSE.}
    \item Hazard detection is also possible without knowing the DEM of the surface. \textbf{TRUE. See 9-8-2 of the reader.}
    \item For landing on Mars, main parachutes trailing at least five meters behind the descent module are not exposed to its wake. \textbf{FALSE?}
    \item For entry from Low Earth Orbit, a re-entry vehicle is not exposed to rarefied gas flow. \textbf{FALSE.}
    \item The gravity turn cannot be used for a precision landing. \textbf{TRUE.}
    \item An inherently safe landing region does not contain any slopes. \textbf{FALSE.}
    \item In the altitude-velocity plane, the heat-flux constraint moves downwards with increasing flux value. \textbf{TRUE?}
    \item An Inertial Measurement Unit operates well during the so-called "black-out" phase. \textbf{TRUE.}
    \item For Lunar entry, on average thicker heat shields than for Mars are required. \textbf{FALSE.}
    \item Active sensors for hazard detection are superior to passive sensors in every aspect. \textbf{FALSE.}
\end{enumerate}
